\documentclass{article}
\usepackage[utf8]{inputenc}
\usepackage{amsmath}
\usepackage{url}
\usepackage{amssymb}
\DeclareMathOperator{\SBM}{SBM}
\title{spectral clustering for SSBM}
\author{zhaof17 }
%\date{April 2021}

\begin{document}
\maketitle

For random graph generated by $\SBM(n,2,p,q)$ in which the first community
has size $\alpha n$ (which is a natural number). $\mathbb{E}[A]$ has rank 2 and its
two non-zero eigenvalues are given by
\begin{equation}\label{eq:lambda}
    \lambda_{1,2} = \frac{n}{2}(p \pm \sqrt{p^2-4\alpha(1-\alpha)(p^2-q^2)})
\end{equation}
or equivalently, $\lambda$ satisfies
$ \lambda^2 - np\lambda + \alpha(1-\alpha)(p^2-q^2) =0$.
The eigenvector is $(1,\dots, 1, \beta, \dots, \beta)$ where
$\beta$ satisfies
$\lambda = n(\alpha p + (1-\alpha)q\beta)$.
This conclusion is mentioned in \cite{yun14}.
When $\alpha = \frac{1}{2}$, it becomes the well-known conclusion $\lambda_{1,2}=\frac{n}{2}(p\pm q)$.

Equation \ref{eq:lambda} can be transformed to
the expression about $\beta$: $(1-\alpha)\beta^2-\frac{p}{q}(1-2\alpha)\beta - \alpha=0$.

When we consider the Laplacian: $L=D-A$. Then
the expectation of $L$ has four eigenvalues, the smallest is 0,
the second smallest is $nq$
while others are $n(p(1-\alpha) + q\alpha)$
or $n(q(1-\alpha) + p\alpha)$.
The second smallest eigenvalue corresponds to eigenvector
$(1, \dots, 1, \beta, \dots, \beta)$ where $\beta=-\frac{\alpha}{1-\alpha}$. Let $\varphi$
be the normalized version of this eigenvector.

Below we need to study when there is a perturbation on $\mathbb{E}[L]$,
how $\varphi_2$ is perturbed such that its sign is not changed.
By Davis-Kahan theorem, the perturbation of $\varphi$
is controlled by
$ |\tilde{\varphi}_i - \varphi_i |
\leq \frac{2^{3/2} ||X||}{\delta}$ for any $i$ \cite{math236-16}.
The random matrix $X$ is $L-\mathbb{E}[L]$, and 
$\delta = \min_{j\neq 2}|\lambda_2(\mathbb{E}[L]) - \lambda_j(\mathbb{E}[L])| = \min\{\alpha, 1-\alpha\} n(p-q)$.
Without loss of generality we can assume $\alpha \geq \frac{1}{2}$. Then we have
\begin{equation}
    |\tilde{\varphi}_i - \varphi_i |
\leq \frac{2^{3/2} ||X||}{(1-\alpha)n(p-q)}
\end{equation}

When $\alpha=\frac{1}{2}$ and $p,q=\Omega(\frac{\log n}{n})$, we can also use the
second largest eigenvalue of
$A$ to recover the node label
with misclassified nodes up to $o(n)$ \cite{mossel}.

We can also deal with the matrix $B=I_n-J_n+2A$,
and use the eigenvector corresponds to the largest eigenvalue to
achieve weak recovery of SBM with two equal-sized communities.
The analysis is the same with that of $L$.

\bibliographystyle{plain}
\begin{thebibliography}{9}
\bibitem{yun14} Yun, Se-Young, and Alexandre Proutiere. "Community detection via random and adaptive sampling." Conference on learning theory. PMLR, 2014.
\bibitem{math236-16} Spectral Clustering, Stochastic Block Modeland Matrix Perturbation \url{https://cims.nyu.edu/~sling/MATH-SHU-236-2020-SPRING/MATH-SHU-236-Lecture-16-SBM-Spectral.pdf}
\bibitem{mossel}
Mossel, Elchanan, Joe Neeman, and Allan Sly. "Consistency thresholds for the planted bisection model." Proceedings of the forty-seventh annual ACM symposium on Theory of computing. 2015.
\end{thebibliography}
\end{document}

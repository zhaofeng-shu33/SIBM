\documentclass{ctexart}
\begin{document}
The label propagation algorithm for
community detection has some similarity with
Jacobi or Gauss-Seidel iteration method
in numerical analysis.

To simplify the discussion, we consider an undirected unweighted graph, whose Laplacian matrix $L$ is a
diagonal dominant matrix. The graph has two balanced underlying community structure,
and the label of each node belongs to $\{\pm\}$.
We can decompose $L=D+(L-D)$.
And use $x^{n+1} = \mathrm{sgn}(D^{-1}(D-L)x^n)$ to update the label.
e consider 
Initially $x^0$ is initialized randomly. The $\mathrm{sgn}$
function is modified in such a way that $\mathrm{sgn}(0)$ is randomly
assigned to $+1$ or $-1$. This scheme is similar to Jacobi iteration method,
which is easy to parallelize. On the other hand, we can use the newly obtained $x^{n+1}[1:k]$
to update $x^{n+1}[k+1]$, which is similar to Gauss-Seidel method, limited to sequence usage.

\end{document}

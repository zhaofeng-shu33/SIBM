\documentclass{article}
\usepackage{amsmath}
\usepackage{algorithm}
\usepackage{algorithmic}
\begin{document}
	\section{Introduction}
	\section{Stochastic Ising BlockModel}\label{sec:sibm}
	Let $W=\{1, \omega, \dots, \omega^{k-1}\}$ be a cyclic group with order $k$.
	Define an Ising model on a graph $G=([n],E(G))$ with parameters $\alpha,\beta>0$
	as the probability distribution on the configurations $\sigma\in W^n$ such that
	\begin{equation} \label{eq:isingma}
	P_{\sigma|G}(\sigma=\bar{\sigma})=\frac{1}{Z_G(\alpha,\beta)}
	\exp\Big(\beta\sum_{\{i,j\}\in E(G)} I(\bar{\sigma}_i ,\bar{\sigma}_j)
	-\frac{\alpha\log(n)}{n} \sum_{\{i,j\}\notin E(G)} I(\bar{\sigma}_i, \bar{\sigma}_j)\Big) ,
	\end{equation}
	where the subscript in $P_{\sigma|G}$ indicates that the distribution depends on $G$, and 
	$Z_G(\alpha,\beta)$
	is the normalizing constant. The indicator function is defined as
	$$
	I(x ,y) = \begin{cases}
	k-1 & x = y\\
	-1 & x \neq y
	\end{cases}
	$$
	
	\section{Metropolis sampling algorithm for SIBM}
	In this section, we give an algorithm based on SIBM to do community detection tasks.
	For a SBM with $k-$ communities, in previous section we have shown that the sample of SIBM is the same with the label $X$ up to some permutation. 
	Therefore, if we can get a sample from SIBM, then we can use this sample as the estimator of $X$.
	
	The difficulty is that we cannot generate the sample exactly from its distribution \eqref{eq:isingma}. Therefore, some approximation is
	necessary. The most common way to generate a Ising sample is using Metropolis sampling \cite{metropolis1953equation}.
	In this article, we use Metropolis sampling to generate SIBM sample.
	
	From equation \eqref{eq:isingma}, we know that for $k$-state Ising model, the Hamiltonian can be written as:
	\begin{equation}
	H(\bar{\sigma}) = \frac{\alpha}{\beta} \frac{\log n}{n} \sum_{\{i,j\}\not\in E(G)} I(\bar{\sigma}_i, \bar{\sigma}_j)
		- \sum_{\{i,j\}\in E(G)} I(\bar{\sigma}_i, \bar{\sigma}_j)
	\end{equation}
	Suppose one coordinate $\bar{\sigma}_r$ flips to $w^s \cdot \bar{\sigma}_r$, the new state is denoted as $\bar{\sigma}^{(\sim \{r\}, v)}$.
	Then the change of energy is:
	\begin{equation}\label{eq:delta_H}
	 H(\bar{\sigma}^{(\sim \{r\}, [\omega^s])}) - H(\bar{\sigma}) = k \sum_{i \in N_r(G)} J_s(\bar{\sigma}_r, \bar{\sigma}_i) - k\frac{\alpha \log n}{\beta n} \sum_{i \in V\backslash N_r(G)} J_s(\bar{\sigma}_r,\bar{\sigma}_i)
	\end{equation}
	where $J_s(\sigma_i, \sigma_j)$ is defined as:
	\begin{equation}
	J_s(\sigma_i, \sigma_j) = \begin{cases}
	1 & \sigma_i = \sigma_j \\
	-1 & \omega^s \sigma_i = \sigma_j \\
	0 & \textrm {otherwise}
	\end{cases}
	\end{equation}
	The pseudo code of our algorithm is summarized in \ref{alg:m}. This algorithm requires the number of the communities $k$ to be known and the weight parameter $\alpha, \beta$ is given. The iteration time $N$ should also be specified.
	\begin{algorithm}[H]
		\caption{Metropolis sampling algorithm for SIBM} \label{alg:m}
		Inputs: the graph $G$ \\
		Output: $\hat{X}$
		\begin{algorithmic}[1]
			\STATE random initialize $\sigma \in W^n$
			\WHILE{the iteration time $<N$}
			\FOR{each node $r$}
			\STATE random choose a new flipping state $w^s \cdot \sigma_r$
			\STATE compute $\Delta H(r,s) = H(\bar{\sigma}^{(\sim \{r\}, [w^s])}) - H(\bar{\sigma})$ using Equation \eqref{eq:delta_H}
			\IF{$\Delta H(r,s)<0$}
			\STATE $\sigma_r \leftarrow w^s \cdot \sigma_r$
			\ELSE
			\STATE generate a random number $u$ ranged within $[0,1]$
			\IF{$u < \exp(-\beta H(r,s))$}
			\STATE $\sigma_r \leftarrow w^s \cdot \sigma_r$
			\ENDIF
			\ENDIF
			\ENDFOR
			\ENDWHILE
		\end{algorithmic}
	\end{algorithm}
	Since the computation of $\Delta H(r,s)$ needs $O(n)$ time. The time complexity of algorithm \ref{alg:m} is $Nn^2$.
	\section{Parameter Estimation for SBM}
	In Section \ref{sec:sibm}, we have shown that when $\alpha > b \beta$ and $\beta > \beta^*$. The sample of SIBM aligns with $X$ with probability one
	as $n\to \infty$. 
	% conjecture, both alpha and beta cannot be too large, otherwise the convergence of Metropolis sampling is very slow.
	The critical value $\beta^*$ is a function of $a$ and $b$. Therefore, to give good guarantee of Algorithm \ref{alg:m}, we need to estimate the parameter
	of $a,b$ before estimating the label of $X$.
	This estimation is done by counting the edges and triangles of the input graph $G$.
	\bibliographystyle{plain}
	\bibliography{exportlist.bib}
\end{document}
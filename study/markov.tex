\documentclass{article}
\usepackage{circledsteps}
\usepackage{amsmath}
\title{Markov and Ising model}
\begin{document}
\maketitle
We can sample from a distribution using Metropolis algorithm. The underlining mechanism is
based on Markov chain. We consider a complete graph with 3 nodes for an example.
Each node takes value from $\{\pm 1\}$.
\Circled{1} $(1,1,1)$,
\Circled{2} $(1,1,-1)$,\Circled{3} $(1,-1,1)$,\Circled{4} $(-1,1,1)$,
\Circled{5} $(1,-1,-1)$,\Circled{6} $(-1,1,-1)$,\Circled{7} $(-1,-1,1)$,
\Circled{8} $(-1,-1,-1)$.
We use the Ising model to describe the probabilty of each state
$$
P(\sigma) = \frac{1}{Z} \exp(\beta \sum_{(i,j) \in E(G)} \sigma_i \sigma_j)
$$
Then for \Circled{1}, \Circled{8}, the probability is
$p_1 = \frac{e^{3\beta}}{2e^{3\beta} + 6 e^{-\beta}}$.
For other six states, the probability is 
$p_2 = \frac{e^{-\beta}}{2e^{3\beta} + 6 e^{-\beta}}$.
We use Metropolis algorithm with random flip to generate this distribution with 8 states,
the transition matrix $P$ is  $8\times 8$ and can be written as follows:
$$
\begin{pmatrix}
1-e^{-4\beta} & \frac{1}{3} e^{-4\beta} & \frac{1}{3} e^{-4\beta} & \frac{1}{3} e^{-4\beta} & 0 & 0 & 0 & 0 \\
\frac{1}{3} & 0 & 0 & 0 & \frac{1}{3} & \frac{1}{3} & 0 & 0 \\
\frac{1}{3}& 0 & 0 & 0 & \frac{1}{3} & 0 & \frac{1}{3} & 0 \\
\frac{1}{3} & 0 & 0 & 0 & 0 & \frac{1}{3} & \frac{1}{3} & 0 \\
0 & \frac{1}{3} & \frac{1}{3} & 0 & 0 & 0 & 0 & \frac{1}{3} \\
0 & \frac{1}{3} & 0 & \frac{1}{3} & 0 & 0 & 0 & \frac{1}{3} \\
0 & 0 & \frac{1}{3} & \frac{1}{3} & 0 & 0 & 0 & \frac{1}{3} \\
0 & 0 & 0 & 0 &  \frac{1}{3} e^{-4\beta} & \frac{1}{3} e^{-4\beta} & \frac{1}{3} e^{-4\beta} & 1-e^{-4\beta}
\end{pmatrix}
$$
The steady state satisfies the equation $\pi P = \pi$ (where $\pi$ is a row vector).
We can verify that $\pi = (p_1, p_2, p_2, p_2, p_2, p_2, p_2, p_1)$ is the solution.
Acutally it satisifes the detailed balance condition: $\pi_i P_{ij} = \pi_j P_{ji}$.

We can verify the above conclusion numerically by choosing $\beta = 0.1$, solving
the eigenvector of $P^T$ givens that the eigenvector 1 corresponds to eigenvector $\pi^T$.

\end{document}
